%================================================================
\chapter{Conclusions and Outlook}
\chaptermark{Conclusions and Outlook}
%================================================================


In this final chapter,...


%================================
\section{Conclusions}
Why is it that so many experiments tend to favor the Minkowski form and that there are only a couple claims to Abraham? The reason lies with the Abraham force $f^A=\varepsilon_0\left(\mathbf{\varepsilon_r}-1\right)\frac{\partial}{\partial t}\mathbf{E}\times\mathbf{B}$. As we saw in Section \ref{sec:forces}, this only becomes relevant in certain situations.  We saw that this force averages to zero over an optical cycle. This makes it notoriously difficult to see in experimentation. The momentum transferred however, is only dependent on the initial and final amplitude of the field.  If one considers the momentum gained from the field due to a field which is initially at zero and is then turned up to some maximal strength $E_{\mathrm{max}}$, one finds the Abraham contribution is twice as much as that due to the gradient force.  Of course for a pulsed field, the Abraham force will contribute zero total momentum as the initial and final amplitudes are both the same.  The reason why the Abraham force has been so elusive, is because we have been doing the wrong experiments. Recently, Rikken and Tiggelen [2] claim to have seen the Abraham force.  In their experiment however, they used a static magnetic field and applied an alternating voltage perpendicular to the magnetic field.  As the Abraham force is applied at a normal angle to the $E$ and $B$ fields, they only measured the force acting in this direction and therefore were able to avoid the issue of having the Abraham force masked by the gradient force.  What we are looking for however is an example in which one observes the Abraham momentum in an electromagnetic field.  Let us first go through some of the big experiments in which the authors make claims to the Minkowski form.
\\
The first experiment to consider was performed by Pritchard and Ketterle \cite{ketterle}.  They used an elongated $^{87}Rb$ Bose-Einstein condensate contained in a magnetic trap.  They used a $\lambda=780 nm$ standing wave pulse which acted for $5 \mu s$ to out couple approximately $5\%$ of the atoms.  After waiting $600 \mu s$ a second pulse was applied which out coupled another group of atoms, interfering with the first group.  Using ballistic imaging they were able to resolve the momentum states and conclude that the atoms had acquired a momentum kick proportional to the refractive index of the gas, thus corroborating Minkowki's claim.  We immediately see the issue here however.  By pulsing the standing wave, they have ensured that the Abraham force will contribute nothing after the full pulse cycle.  

So what is the message to take home from this?  Let's go back and take a look at Poynting's theorem in the Minkowski and the Abraham representation.  In the Minkowski representation we have
\begin{equation}
\mathbf{f}+\frac{\partial}{\partial t}\left[\mathbf{D}\times\mathbf{B}\right]=\nabla\cdot\left(\mathbf{E}\mathbf{D}+\mathbf{H}\mathbf{B}-\frac{1}{2}\mathbf{I}\left(\mathbf{D}\cdot\mathbf{E}+\mathbf{H}\cdot\mathbf{B}\right)\right)
\end{equation}
\\
where we identify $\mathbf{f}$ with the mechanical momentum of system.  What this equation is telling us is that 
\begin{equation}
\frac{\partial}{\partial t}\left(\mathbf{P}_{\mathrm{mechanical}}\right)+\frac{\partial}{\partial t}\left(\mathbf{P}_{\mathrm{electromagnetic}}\right)=\nabla\cdot\mathbf{W}_{\mathrm{total}}
\end{equation}
where $\mathbf{P}_{\mathrm{mechanical}}$ is the mechanical momentum of the material, $\mathbf{P}_{\mathrm{electromagnetic}}$ is the electromagnetic momentum, and $\mathbf{W}_{\mathrm{total}}$ is the total work done.  Now just as we had previously done in the Energy Momentum section, we and subtract  
$\varepsilon_0\left(\mathbf{\varepsilon_r}+1\right)\frac{\partial}{\partial t}\mathbf{E}\times\mathbf{B}$ from both sides.  This gives us
\begin{equation}
\tilde{\mathbf{f}}+\frac{\partial}{\partial t}\frac{\mathbf{E}\times\mathbf{H}}{c^2} =\nabla\cdot\left(\mathbf{E}\mathbf{D}+\mathbf{H}\mathbf{B}-\frac{1}{2}\mathbf{I}\left(\mathbf{D}\cdot\mathbf{E}+\mathbf{H}\cdot\mathbf{B}\right)\right)
\end{equation}
where $\tilde{\mathbf{f}}=\mathbf{f}+\varepsilon_0\left(\mathbf{\varepsilon_r}+1\right)\frac{\partial}{\partial t}\mathbf{E}\times\mathbf{B}$.  Now in the Abraham representation, all we have done is identified $\tilde{\mathbf{f}}$ with $\frac{\partial}{\partial t}\mathbf{P}_{\mathrm{mechanical}}$ and $\frac{\mathbf{E}\times\mathbf{H}}{c^2}$ with $\frac{\partial}{\partial t}\mathbf{P}_{\mathrm{electromagnetic}}$.  So how does this fit in with our findings that the Abraham force $\varepsilon_0\left(\mathbf{\varepsilon_r}+1\right)\frac{\partial}{\partial t}\mathbf{E}\times\mathbf{B}$ is usually negligible?  Consider that experiments will measure the momentum impulse $\tilde{\mathbf{f}}$ of the atoms and since the Abraham term is so notoriously difficult to detect, they will detect the absence of this term and will therefore assume the material momentum is $\mathbf{f}$ which will lead them to the Minkowski momentum.  So what is the correct picture.  The correct picture is the Abraham representation.  Going back to the derivation of the Minkowski representation, it's clear that the material momentum only considers the free charge momentum.  The Abraham representation correctly accounts for the momentum of the bound charges which travel along with the pulse.  These bound charges are pulled towards the pulse as it approaches, and then rapidly brought to a halt as the pulse passes.  It is this ephemeral motion which is to blame for all the confusion.  




%================================
\section{Future Directions}





%================================================================
